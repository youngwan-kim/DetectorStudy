
\section{Interactions of Particles and Radiation with Matter}
    
\begin{problem}{1.1}
The range of a 100 keV electron in water is about 200 $\unit{\micro\meter}$. Estimate its stopping time.
\end{problem}
\begin{solution}

\end{solution}

\noindent\rule{7in}{1.5pt}

%%%%%%%%%%%%%%%%%%%%%%%%%%%%%%%%%%%%%%%%%%%%%%%%%%%%%%%%%%%%%%%%%%%%%%%%%%%%%%%%%%%%%%%%%%%%%%%%%%%%%%%%%%%%%%%%%%%%%%%%%%%%%%%%%%%%%%%%

\begin{problem}{1.2}
The energy loss of TeV muons in rock can be parametrised by

\begin{align*}
    - \frac{\dif E}{\dif x} = a + bE
\end{align*}\\
where $a$ stands for a parametrisation of the ionisation loss and the $b$ term includes bremsstrahlung, direct electron-pair production and nuclear interactions $\left( a\approx 2\unit{ MeV}/(\unit{g/cm^2}), b=\num{4.4e-6}(\unit{g/cm^2})^{-1} \right)$. Estimate the range of a 1 TeV muon in rock.
\end{problem}
\begin{solution}

\end{solution}

\noindent\rule{7in}{1.5pt}

%%%%%%%%%%%%%%%%%%%%%%%%%%%%%%%%%%%%%%%%%%%%%%%%%%%%%%%%%%%%%%%%%%%%%%%%%%%%%%%%%%%%%%%%%%%%%%%%%%%%%%%%%%%%%%%%%%%%%%%%%%%%%%%%%%%%%%%%

\begin{problem}{1.3}
    Monoenergetic electrons of 500keV are stopped in a silicon counter. Work out the energy resolution of the semiconductor detector if a Fano factor of 0.1 at 77 K is assumed.
\end{problem}
\begin{solution}

\end{solution}

\noindent\rule{7in}{1.5pt}

%%%%%%%%%%%%%%%%%%%%%%%%%%%%%%%%%%%%%%%%%%%%%%%%%%%%%%%%%%%%%%%%%%%%%%%%%%%%%%%%%%%%%%%%%%%%%%%%%%%%%%%%%%%%%%%%%%%%%%%%%%%%%%%%%%%%%%%%

\begin{problem}{1.4}
For non-relativistic particles of charge z the Bethe-Bloch formula can be approximated by

\begin{align*}
    - \frac{\dif E_{\text{kin}}}{\dif x} = a \frac{z^2}{E_{\text{kin}}} \ln\left(bE_{\text{kin}}\right),
\end{align*}\\
where $a$ and $b$ are material-dependent constants (different from those in Problem 1.2). Work out the energy-range relation if $ \ln\left(bE_{\text{kin}}\right)$ can be approximated by $\left(bE_{\text{kin}}\right)^{1/4}$.
\end{problem}
\begin{solution}

\end{solution}

\noindent\rule{7in}{1.5pt}

%%%%%%%%%%%%%%%%%%%%%%%%%%%%%%%%%%%%%%%%%%%%%%%%%%%%%%%%%%%%%%%%%%%%%%%%%%%%%%%%%%%%%%%%%%%%%%%%%%%%%%%%%%%%%%%%%%%%%%%%%%%%%%%%%%%%%%%%

\begin{problem}{1.5}
In \textit{Compton telescopes} for astronomy or medical imaging one frequently needs the relation between the scattering angle of the electron and that of the photon/ Work out this relation from momentum conservation in the scattering process.
\end{problem}
\begin{solution}

\end{solution}

\noindent\rule{7in}{1.5pt}

%%%%%%%%%%%%%%%%%%%%%%%%%%%%%%%%%%%%%%%%%%%%%%%%%%%%%%%%%%%%%%%%%%%%%%%%%%%%%%%%%%%%%%%%%%%%%%%%%%%%%%%%%%%%%%%%%%%%%%%%%%%%%%%%%%%%%%%%

\begin{problem}{1.6}
The ionisation trail of charged particles in a gaseous detector is mostly produced by low-energy electrons. Occasionally, a larger amount of energy can be transferred to electrons ($\delta$ rays, knock-on electrons). Derive the maximum energy that a 100 GeV muon can transfer to a free electron at rest in a $\mu e$ collision.
\end{problem}
\begin{solution}

\end{solution}

\noindent\rule{7in}{1.5pt}

%%%%%%%%%%%%%%%%%%%%%%%%%%%%%%%%%%%%%%%%%%%%%%%%%%%%%%%%%%%%%%%%%%%%%%%%%%%%%%%%%%%%%%%%%%%%%%%%%%%%%%%%%%%%%%%%%%%%%%%%%%%%%%%%%%%%%%%%

\begin{problem}{1.7}
The production of $\delta$ rays can be described by the Bethe-Bloch formula. To good approximation the probability for $\delta$-ray production is given by

\begin{align*}
    \phi(E) \dif E = K \frac{1}{\beta^2} \frac{Z}{A} \cdot \frac{x}{E^2} \dif E,
\end{align*}\\
where

\begin{align*}
    K &= 0.154 \unit{MeV}/(\unit{g/cm^2}),\\
    Z,A &= \text{atomic number and mass of the target},\\
    x &= \text{absorber thickness in } \unit{g/cm^2},\\
\end{align*}
Work out the probability that a 10 GeV muon produces a $\delta$ ray of more than $E_0 = 10$ MeV in an 1 cm argon layer (gas at standard room temperature and pressure).
\end{problem}
\begin{solution}

\end{solution}

\noindent\rule{7in}{1.5pt}

%%%%%%%%%%%%%%%%%%%%%%%%%%%%%%%%%%%%%%%%%%%%%%%%%%%%%%%%%%%%%%%%%%%%%%%%%%%%%%%%%%%%%%%%%%%%%%%%%%%%%%%%%%%%%%%%%%%%%%%%%%%%%%%%%%%%%%%%

\begin{problem}{1.8}
Relativistic particles suffer an approximately constant ionisation energy loss of about 2 $\unit{MeV}/(\unit{g/cm^2})$. Work out the depth-intensity relation of cosmic-ray muons in rock and estimate the intensity variation if a cavity of height $\Delta h$ = 1 m at a depth of 100 m were in the muon beam.
\end{problem}
\begin{solution}

\end{solution}

\noindent\rule{7in}{1.5pt}

%%%%%%%%%%%%%%%%%%%%%%%%%%%%%%%%%%%%%%%%%%%%%%%%%%%%%%%%%%%%%%%%%%%%%%%%%%%%%%%%%%%%%%%%%%%%%%%%%%%%%%%%%%%%%%%%%%%%%%%%%%%%%%%%%%%%%%%%
